%%%%%%%%%%%%%%%%%%%%%%%%%%%%%%%%%%%%%%%%%
% University Assignment Title Page 
% LaTeX Template
% Version 1.0 (27/12/12)
%
% This template has been downloaded from:
% http://www.LaTeXTemplates.com
%
% Original author:
% WikiBooks (http://en.wikibooks.org/wiki/LaTeX/Title_Creation)
%
% License:
% CC BY-NC-SA 3.0 (http://creativecommons.org/licenses/by-nc-sa/3.0/)
% 
% Instructions for using this template:
% This title page is capable of being compiled as is. This is not useful for 
% including it in another document. To do this, you have two options: 
%
% 1) Copy/paste everything between \begin{document} and \end{document} 
% starting at \begin{titlepage} and paste this into another LaTeX file where you 
% want your title page.
% OR
% 2) Remove everything outside the \begin{titlepage} and \end{titlepage} and 
% move this file to the same directory as the LaTeX file you wish to add it to. 
% Then add \input{./title_page_1.tex} to your LaTeX file where you want your
% title page.
%
%%%%%%%%%%%%%%%%%%%%%%%%%%%%%%%%%%%%%%%%%
%\title{Title page with logo}
%----------------------------------------------------------------------------------------
%	PACKAGES AND OTHER DOCUMENT CONFIGURATIONS
%----------------------------------------------------------------------------------------

\documentclass[12pt, a4paper]{article}
\usepackage[english]{babel}
\usepackage[utf8]{inputenc}
\usepackage{amsmath}
\usepackage{graphicx}
\usepackage{subfig}
\usepackage[colorinlistoftodos]{todonotes}

\usepackage{footnote}
\makesavenoteenv{figure}

\usepackage[babel,english=british]{csquotes} % cool quotes
\usepackage[backend=biber]{biblatex} % bibliogrpahy
\addbibresource{literature.bib}

\begin{document}

\begin{titlepage}

\newcommand{\HRule}{\rule{\linewidth}{0.5mm}} % Defines a new command for the horizontal lines, change thickness here

\center % Center everything on the page
 
%----------------------------------------------------------------------------------------
%	HEADING SECTIONS
%----------------------------------------------------------------------------------------

\textsc{\LARGE Ruprecht-Karls-Universität Heidelberg}\\[1.5cm] % Name of your university/college
\textsc{\Large Master Pflichtseminar}\\[0.5cm] % Major heading such as course name
\textsc{\large Physik moderner MRT / CT Techniken}\\[0.5cm] % Minor heading such as course title

%----------------------------------------------------------------------------------------
%	TITLE SECTION
%----------------------------------------------------------------------------------------

\HRule \\[0.4cm]
{ \huge \bfseries fMRT: BOLD-Kontrast und Darstellung von Gehirnaktivierung}\\[0.4cm] % Title of your document
\HRule \\[1.5cm]
 
%----------------------------------------------------------------------------------------
%	AUTHOR SECTION
%----------------------------------------------------------------------------------------

\begin{minipage}{0.4\textwidth}
\begin{flushleft} \large
\emph{Author:}\\
Lucas-Raphael \textsc{Müller} % Your name
\end{flushleft}
\end{minipage}
~
\begin{minipage}{0.4\textwidth}
\begin{flushright} \large
\emph{Supervisor:} \\
Michael \textsc{Ruttorf} % Supervisor's Name
\end{flushright}
\end{minipage}\\[2cm]

% If you don't want a supervisor, uncomment the two lines below and remove the section above
%\Large \emph{Author:}\\
%John \textsc{Smith}\\[3cm] % Your name

%----------------------------------------------------------------------------------------
%	DATE SECTION
%----------------------------------------------------------------------------------------

{\large \today}\\[2cm] % Date, change the \today to a set date if you want to be precise

%----------------------------------------------------------------------------------------
%	LOGO SECTION
%----------------------------------------------------------------------------------------

% \includegraphics{logo.png}\\[1cm] % Include a department/university logo - this will require the graphicx package
 
%----------------------------------------------------------------------------------------

\vfill % Fill the rest of the page with whitespace

\end{titlepage}

\begin{abstract}
Your abstract.
\end{abstract}

\tableofcontents

\section{Introduction}
\label{sec:intro}

\subsection{Outline}
\subsection{What's the aim?}

\section{Physiology}
Not only is the knowledge of the functional structure of the human brain a major issue for neuroscientists, it's vast complexity is also fascinating for almost everyone.
Major parts of the brain consists of neurons. 
Biophysical aspects are understood on a neuron-wise context, membrane potentials, ion channels and signal propagation are topics to be found in all physiology books which are worth the paper they have been printed on.\cite[577 et. seq.]{guyton}
% Commands to include a figure:

\begin{figure}[ht]
   \centering
      %\subfloat[CAPTION]{BILDERCODE}\qquad
      \subfloat[Structure of a large neuron in the brain showing it's important functional parts. \protect\cite{guyton}]{\includegraphics[scale=0.2]{pictures/largeNeuron.png}}\qquad %page 578
      \subfloat[Top row: hypothetical plots of average neuronal activity over time. Bottom row: corresponding functional magnetic resonance imaging (fMRI) responses. Left: hypothetical haemodynamic impulse response function (HIRF) measured as the response to a brief pulse of neuronal activity. Right: the fMRI response when the average neuronal activity alternates (at specific times) between three different states.\protect\cite{Heeger2002}]{\includegraphics[width = 0.5\textwidth]{pictures/signalProcessing.png}}
   %\caption[(a) bla]
\end{figure}

Being aware that a physical description is possible on a microscopic scale, it is a task of huge complexity to describe neural activity in total.
Direct measurement of neural activity, i.e. placing eletrodes directly in the brain, requires surgery and most of the time this can only be done in animal experiments. 
Information can be received at the scalp of the brain, however this is a rather rough estimation and not useful for drawing precise activation maps.\cite[6]{buxton}
Modern approaches are positron emission tomography (PET) or funtional magnetic resonance imaging among others.
These methods rely on a implicit measurement, as no electrical field or voltage drop is measured, but other quantities which are followup-like, namely glucose consumption or blood oxygenation level, which are again measured implicitly.
In the former case, tracers with glucose accumulate in regions of high glucose consumption resulting in a local higher active radiation source.
In the latter case, blood oxygenation level changes local inhomogenities of the magnetic field and leads to different \textit{T2*} relaxation.
The author wants to focus on aspects of the latter one and shortly describe important underlying principles as well as the most basic but necessary physiological knowledge.

\subsection{Neural Activity}





Notes:
Deoxy is paramagnetic, induces magnetic field inhomogenity \\
Oxy is diamagnetic, no influence \\
Cerebal blood flow. Brain activity -. blood flow -. oxygenated hemoglobin -. $T_2^*$ -. MRI signal

\section{Hardware}
\label{sec:hardware}

\section{Sequence and Signal}
\label{sec:sequenceSignal}

\section{Software}
\label{sec:software}

\section{Real example (me)}
\label{sec:example}

\printbibliography

\end{document}


\subsection{Sections}

Use section and subsection commands to organize your document. \LaTeX{} handles all the formatting and numbering automatically. Use ref and label commands for cross-references.

\subsection{Comments}

Comments can be added to the margins of the document using the \todo{Here's a comment in the margin!} todo command, as shown in the example on the right. You can also add inline comments too:

\todo[inline, color=green!40]{This is an inline comment.}

\subsection{Tables and Figures}

Use the table and tabular commands for basic tables --- see Table~\ref{tab:widgets}, for example. You can upload a figure (JPEG, PNG or PDF) using the files menu. To include it in your document, use the includegraphics command as in the code for Figure~\ref{fig:frog} below.

\subsection{Mathematics}

% Commands to include a figure:
\begin{figure}
\centering
\includegraphics[width=0.5\textwidth]{frog.jpg}
\caption{\label{fig:frog}This is a figure caption.}
\end{figure}

\begin{table}
\centering
\begin{tabular}{l|r}
Item & Quantity \\\hline
Widgets & 42 \\
Gadgets & 13
\end{tabular}
\caption{\label{tab:widgets}An example table.}
\end{table}
